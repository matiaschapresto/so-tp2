\documentclass[a4paper,10pt]{article}
\usepackage[paper=a4paper, hmargin=1.5cm, bottom=1.5cm, top=3.5cm]{geometry}
\usepackage[latin1]{inputenc}
\usepackage[T1]{fontenc}
\usepackage[spanish]{babel}
\usepackage{amssymb}
\usepackage{amsmath}
\usepackage{mathtools}
\usepackage{fancyhdr}
\usepackage{lastpage}
\usepackage{caratula}
\usepackage{verbatim}
\usepackage{xspace}
\usepackage{xargs}
\usepackage{float}
\usepackage{graphicx}
\usepackage{ifthen}
\usepackage[spanish,noline,longend]{algorithm2e}
\usepackage{listings}
\usepackage{braket}
\usepackage{color}

\definecolor{mygreen}{rgb}{0,0.6,0}
\definecolor{mygray}{rgb}{0.5,0.5,0.5}
\definecolor{mymauve}{rgb}{0.58,0,0.82}

%\usepackage{aed2-tad,aed2-symb,aed2-itef}

\lstset{language=C++, tabsize=4, breaklines=true, breakatwhitespace=true, numbers=left, numbersep=10pt}

\newcommand{\moduloNombre}[1]{\textbf{#1}}

\let\NombreFuncion=\textsc
\let\TipoVariable=\texttt
\let\ModificadorArgumento=\textbf
\newcommand{\res}{$res$\xspace}
\newcommand{\tab}{\hspace*{7mm}}

\newcommandx{\TipoFuncion}[3]{%
  \NombreFuncion{#1}(#2) \ifx#3\empty\else $\to$ \res\,: \TipoVariable{#3}\fi%
}
\newcommandx{\Pre}[1][1=true]{\textbf{Pre} $\equiv$ \{#1\}\\}
\newcommand{\Post}[1]{\textbf{Post} $\equiv$ \{#1\}}
\newcommand{\In}[2]{\ModificadorArgumento{in} \ensuremath{#1}\,: \TipoVariable{#2}\xspace}
\newcommand{\Out}[2]{\ModificadorArgumento{out} \ensuremath{#1}\,: \TipoVariable{#2}\xspace}
\newcommand{\Inout}[2]{\ModificadorArgumento{in/out} \ensuremath{#1}\,: \TipoVariable{#2}\xspace}
\newcommand{\Aplicar}[2]{\NombreFuncion{#1}(#2)}

\newlength{\IntFuncionLengthA}
\newlength{\IntFuncionLengthB}
\newlength{\IntFuncionLengthC}
%InterfazFuncion(nombre, argumentos, valor retorno, precondicion, postcondicion, complejidad, descripcion, aliasing)
\newcommandx{\InterfazFuncion}[9][4=true,6,7,8,9]{%
  \hangindent=\parindent
  \TipoFuncion{#1}{#2}{#3}\\%
%  \textbf{Pre} $\equiv$ \{#4\}\\%
%  \textbf{Post} $\equiv$ \{#5\}%
  \Pre[#4]
  \Post{#5}
  \ifx#6\empty\else\\\textbf{Complejidad:} #6\fi%
  \ifx#7\empty\else\\\textbf{Descripci¢n:} #7\fi%
  \ifx#8\empty\else\\\textbf{Aliasing:} #8\fi%
  \ifx#9\empty\else\\\textbf{Requiere:} #9\fi%
}



\newenvironment{Interfaz}{%
  \parskip=2ex%
  \noindent\textbf{\Large Interfaz}%
  \par%
}{}

\newcommand{\Forcond}[2]{
  #1 \textbf{to} #2
}

\newenvironment{Representacion}{%
  \vspace*{2ex}%
  \noindent\textbf{\Large Representaci¢n}%
  \vspace*{2ex}%
}{}

\newenvironment{Algoritmos}{%
  \vspace*{2ex}%
  \noindent\textbf{\Large Algoritmos}%
  \vspace*{2ex}%
}{}

%
%\newcommandx{\Signatura}[3][3]{%
%  \NombreFuncion{#1}(#2)
%  \ifx#3\empty\else $\to$ \res\,: \TipoVariable{#3}\fi
%  \\
%}


\newenvironmentx{algoritmo}[6][3,4,5,6]{
  \begin{algorithm}[H]
  \DontPrintSemicolon
  \newcommandx{\Signatura}[3][3]{
    \NombreFuncion{##1}(##2)
    \ifx##3\empty\else $\to$ \res\,: \TipoVariable{##3}\fi
    \\
  }
  \newcommand{\asignar}{$\leftarrow$ }
  \newcommand{\return}{\textbf{return} }
  \newcommand{\Break}{\textbf{break} }
  \Signatura{#1}{#2}[#3]
  \ifx#4\empty\else\Pre[#4]\fi
  \ifx#5\empty\else\Post{#5}\\\fi
  \ifx#6\empty\else\textbf{Complejidad:} #6\\\fi%
}{\end{algorithm} \vspace{0.3cm}}

\newenvironmentx{algoritmosimple}{
  \begin{algorithm}[H]
  \DontPrintSemicolon
  \newcommand{\asignar}{$\leftarrow$ }
  \newcommand{\return}{\textbf{return} }
  \newcommand{\Break}{\textbf{break} }
}{\end{algorithm} \vspace{0.3cm}}


\newcommand{\Titulon}[1]{
  \vspace*{1ex}\par\noindent\textbf{\large #1}\par
}

\newenvironmentx{Estructura}[2][2={estr}]{%
  \par\vspace*{2ex}%
  \TipoVariable{#1} \textbf{se representa con} \TipoVariable{#2}%
  \par\vspace*{1ex}%
}{%
  \par\vspace*{2ex}%
}%

\newboolean{EstructuraHayItems}
\newlength{\lenTupla}
\newenvironmentx{Tupla}[1][1={estr}]{%
    \settowidth{\lenTupla}{\hspace*{3mm}donde \TipoVariable{#1} es \TipoVariable{tupla}$($}%
    \addtolength{\lenTupla}{\parindent}%
    \hspace*{3mm}donde \TipoVariable{#1} es \TipoVariable{tupla}$($%
    \begin{minipage}[t]{\linewidth-\lenTupla}%
    \setboolean{EstructuraHayItems}{false}%
}{%
    $)$%
    \end{minipage}
}

\newcommandx{\tupItem}[3][1={\ }]{%
    %\hspace*{3mm}%
    \ifthenelse{\boolean{EstructuraHayItems}}{%
        ,#1%
    }{}%
    \emph{#2}: \TipoVariable{#3}%
    \setboolean{EstructuraHayItems}{true}%
}

\newcommandx{\RepFc}[3][1={estr},2={e}]{%
  \tadOperacion{Rep}{#1}{boolean}{}%
  \tadAxioma{Rep($#2$)}{#3}%
}%

\newcommandx{\Rep}[3][1={estr},2={e}]{%
  \tadOperacion{Rep}{#1}{boolean}{}%
  \tadAxioma{Rep($#2$)}{true \ssi #3}%
}%

\newcommandx{\Abs}[5][1={estr},3={e}]{%
  \tadOperacion{Abs}{#1/#3}{#2}{Rep($#3$)}%
  \settominwidth{\hangindent}{Abs($#3$) \igobs #4: #2 $\mid$ }%
  \addtolength{\hangindent}{\parindent}%
  Abs($#3$) \igobs #4: #2 $\mid$ #5%
}%

\newcommandx{\AbsFc}[4][1={estr},3={e}]{%
  \tadOperacion{Abs}{#1/#3}{#2}{Rep($#3$)}%
  \tadAxioma{Abs($#3$)}{#4}%
}%

\let\agregar=\argumento

\newcommand{\DRef}{\ensuremath{\rightarrow}}

\pagestyle{fancy}
\thispagestyle{fancy}
\addtolength{\headheight}{1pt}
\lhead{Sistemas Operativos}
\rhead{$2^{\mathrm{do}}$ cuatrimestre de 2014}
\cfoot{\thepage /\pageref{LastPage}}
\renewcommand{\footrulewidth}{0.4pt}


\titulo{Trabajo Pr\'actico II - Threads}
\fecha{ 12 / 11 / 2014}
\materia{Sistemas Operativos}
\grupo{Grupo n\'umero }
\integrante{Straminsky, Axel}{769/11}{axelstraminsky@gmail.com}
\integrante{Chapresto, Matias}{201/12}{matiaschapresto@gmail.com}
\integrante{Torres, Sebastian}{723/06}{sebatorres1987@hotmail.com}

% Configuration for the code style
\lstset{ %
  backgroundcolor=\color{white},   % choose the background color; you must add \usepackage{color} or \usepackage{xcolor}
  basicstyle=\footnotesize,        % the size of the fonts that are used for the code
  breakatwhitespace=false,         % sets if automatic breaks should only happen at whitespace
  breaklines=true,                 % sets automatic line breaking
  captionpos=b,                    % sets the caption-position to bottom
  commentstyle=\color{mygreen},    % comment style
  deletekeywords={...},            % if you want to delete keywords from the given language
  escapeinside={\%*}{*)},          % if you want to add LaTeX within your code
  extendedchars=true,              % lets you use non-ASCII characters; for 8-bits encodings only, does not work with UTF-8
  frame=single,                    % adds a frame around the code
  keepspaces=true,                 % keeps spaces in text, useful for keeping indentation of code (possibly needs columns=flexible)
  keywordstyle=\color{blue},       % keyword style
  language=Octave,                 % the language of the code
  morekeywords={*,...},            % if you want to add more keywords to the set
  numbers=none,                    % where to put the line-numbers; possible values are (none, left, right)
  numbersep=5pt,                   % how far the line-numbers are from the code
  numberstyle=\tiny\color{mygray}, % the style that is used for the line-numbers
  rulecolor=\color{black},         % if not set, the frame-color may be changed on line-breaks within not-black text (e.g. comments (green here))
  showspaces=false,                % show spaces everywhere adding particular underscores; it overrides 'showstringspaces'
  showstringspaces=false,          % underline spaces within strings only
  showtabs=false,                  % show tabs within strings adding particular underscores
  stepnumber=2,                    % the step between two line-numbers. If it's 1, each line will be numbered
  stringstyle=\color{mymauve},     % string literal style
  tabsize=4,                       % sets default tabsize to 2 spaces
  title=\lstname                   % show the filename of files included with \lstinputlisting; also try caption instead of title
}
\begin{document}

\maketitle
\tableofcontents

\newpage
\section{Introducci\'on}
En el presente trabajo vamos a tratar de resolver el enunciado propuesto por la c\'atedra mediante el uso de \emph{threads}.

\subsection{Objetivo}
El objetivo es implementar un servidor que acepte m\'ultiples conexiones de clientes (usando sockets) y hacer que los mismos se comporten de determinada manera para poder emular la interacci\'on entre alumnos, rescatistas y un aula que debe ser desalojada lo antes posible para practicar un simulacro de incendio, donde debemos tener en cuenta los movimientos de traslado de los estudiantes, ya que hay diversas condiciones.

\subsection{Threads y Pthreads}
Los threads son ampliamente utilizados para implementaci\'on de servidores \textbf{HTTP}, clientes de mensajer\'ia instantanea, UI, entre otras \'areas. Por otro lado, \textbf{Pthreads} es una \textbf{especificaci\'on} y no una librer\'ia. Pthreads indica qu\'e funciones deber\'ia proveer una librer\'ia de threads, sin importar el lenguaje de programaci\'on que se est\'e usando.


Una de las ventajas de usar threads es que todos comparten el mismo conjunto de datos que el thread padre que los crea. Adem\'as son much\'isimo m\'as eficientes que los subprocesos creados utilizando \textbf{\emph{fork()}} en cuando a performance.

Pero a la vez su gran ventaja es su gran desventaja, porque al compartir el mismo conjunto de datos tenemos que tener cuidado cuando los mismos son accedidos o modificados. Comienzan a tomar reelevancia factores como el scheduling, race conditions, entre otros.

En este trabajo adem\'as de implementar el servidor multi con threads, se va a tener que tener especial cuidado a la sincrinizaci\'on de los mismos para que la ejecuci\'on del servidor no tenga comportamientos inesperados.

\newpage
\section{Detalles de implementaci\'on}
\label{sec:Imp}
En esta secci\'on vamos a dar detalle de nuestra implementaci\'on del servidor multicliente. Para sincronizar todos los threads se usan las variables presentes a continuaci\'on:

\lstinputlisting[language=C, frame=single, firstline=20, lastline=32]{../codigo/servidor_multi.c}

A lo largo de la secci\'on vamos a ir justificando la presencia de cada una de ellas.

\subsection{Implementando el servidor \textbf{servidor\_multi.c}}
Para implementar el servidor con soporte para m\'ultiples clientes nos basamos en el presente en \textbf{servidor\_mono.c}, tal y como sugiri\'o la c\'atedra.
La idea principal es que al momento de conectarse el cliente, levantemos un thread con la rutina que ejecutar\'ia el mismo en un servidor mono cliente. A continuaci\'on se muestra el c\'odigo:

\lstinputlisting[language=C, frame=single, lastline=20]{../codigo/servidor_multi.c}

Para poder pasarle par\'ametros a la rutina asociada a los threads, tuvimos que cambiar su interf\'az, es decir, antes la misma era:

\lstinputlisting[language=C, frame=single, firstline=97, lastline=97]{../codigo/servidor_mono.c}

y la cambiamos por:

\lstinputlisting[language=C, frame=single, firstline=216, lastline=216]{../codigo/servidor_multi.c}

Luego de ese cambio de interf\'az el c\'odigo de \textbf{atendedor\_de\_alumno} est\'a preparado para ser llamado desde un thread. Como hab\'iamos mencioado previamente, la idea ser\'ia crear un thread mientras me sigan llegando pedidos de conexi\'on al socket del server.

La funci\'on principal presenta la siguiente implementaci\'on:

\lstinputlisting[language=C, frame=single, firstline=274, lastline=339]{../codigo/servidor_multi.c}

Vemos como en el \textbf{for(;;)} se crean los threads, uno por cada conexi\'on con un cliente. Ac\'a es donde cobra valor la estructura definida previamente, \emph{thread\_args}, que la usamos para pasarle los par\'ametros necesario que la funci\'on \textbf{atendedor\_de\_alumno} necesita para funcionar.

\subsection{Herramientas usadas/cambiadas para realizar pruebas}
Para poder testear la implementaci\'on se hizo un script en \emph{bash} para que corra una cierta cantidad de clientes en simult\'aneo. Adem\'as se hicieron unas ligeras modificaciones al \emph{server\_tester.py} provisto por la c\'atedra. El cambio consiste en hacer que cada vez que se ejecute \emph{server\_tester.py} se elija un nombre de pa\'is distinto. Luego, si ejecutamos 20 clientes en simult\'aneo vamos a tener una visi\'on mas clara de quienes est\'an ejecutando ya que no todos se llaman de la misma manera.

El script de \emph{bash} es el siguiente:

\lstinputlisting[language=bash, frame=single]{../codigo/test_server.sh}
 
 El uso es sencillo, se debe ejecutar pas\'andole como parametro la cantidad de alumnos que queremos que ingresen al aula, o m\'as t\'ecnicamente, la cantidad de clientes que queremos que se conecten al servidor.
 
Por otro lado, en \emph{server\_tester.py} se incluy\'o el modulo \emph{random} al principio para poder obtener un pa\'is al azar de la n-tupla de pa\'ises.
  
\lstinputlisting[language=Python, lastline=13, frame=single]{../codigo/server_tester.py}

La obtenci\'on del nuevo pa\'is para el cliente se hace de la siguiente manera:
\lstinputlisting[language=Python, firstline=86, lastline=89, frame=single]{../codigo/server_tester.py}

La idea de las \'ultimas l\'ineas era crear una posici\'on inicial random para cada cliente creado. El motivo no es otro que poder tener diferentes casos de prueba, y no s\'olo que todos intenten ingresar a la misma posici\'on. Notar que si bien esta implementaci\'on no garantiza que no se repitan nombres de pa\'ises, al menos es mucho mejor que tener que lidiar con m\'ultiples clientes conectados mostrando todos el mismo nombre. Con estas peque\~nas modificaciones se nos facilit\'o mucho el testing del servidor.

% section  (end)

\newpage
\section{Referencias}
\label{sec:Ref}

% section  (end)
\end{document}